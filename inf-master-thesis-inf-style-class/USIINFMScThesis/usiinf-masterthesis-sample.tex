\documentclass[mscthesis]{usiinfthesis}
\usepackage{lipsum}
\usepackage{mathtools}
\usepackage{listings}
\usepackage{subcaption}
\usepackage[most]{tcolorbox}
\usepackage{xcolor,pifont}
\usepackage{color} % for the command \textcolor
\usepackage{soul} % for the command \hl
\newboolean{showcomments}
\setboolean{showcomments}{true}
\ifthenelse{\boolean{showcomments}}
{\newcommand{\nb}[2] {
		\fcolorbox{black}{gray!20}{\bfseries\sffamily\scriptsize#1:}
		{\sf\small$\blacktriangleright$\textit{#2}$\blacktriangleleft$}
	}
}
{\newcommand{\nb}[2]{}
}
\newcommand\matteo[1]{\nb{Matteo}{\hl{#1}}}
\lstset
{
    language=Python,
    breaklines=true,
    basicstyle=\footnotesize,
    frame=single,
    numbers=left,
    stepnumber=1,
    tabsize=1,
    frameround=fttt
}
\usepackage{graphicx}
\usepackage{wrapfig}
\usepackage{multirow}
\graphicspath{{images/}}

\lstdefinelanguage{algebra}
{morekeywords={import,sort,constructors,observers,transformers,axioms,if,
else,end},
sensitive=false,
morecomment=[l]{//s},
}

\title{Training a Real World Reinforcement Learning Agent} %compulsory
\specialization{Major in Artificial intelligence}%optional
\subtitle{A case study using the DonkeyCar autonomous driving framework} %optional 
\author{Giorgio Macauda} %compulsory
\begin{committee}
\advisor{Prof.}{Paolo}{Tonella} %compulsory
\coadvisor{PhD}{Matteo}{Biagiola}{} %optional
\end{committee}
\Day{12} %compulsory
\Month{September} %compulsory
\Year{2022} %compulsory, put only the year
\place{Lugano} %compulsory

% \dedication{To my beloved} %optional
% \openepigraph{Someone said \dots}{Someone} %optional

%\makeindex %optional, also comment out \theindex at the end

\begin{document}

\maketitle %generates the titlepage, this is FIXED

\frontmatter %generates the frontmatter, this is FIXED

\begin{abstract}
\vspace{\fill} 
Reinforcement learning (RL) allows robots to learn skills from interactions with an environment that can be real or simulated, where the problem can be modelled as a sequential decision-making task. In practice, recent advances in RL involve the use of a simulator, where training can be accelerated by parallelizing data collection and safety is not an issue. On the other hand, the field advances at a much slower rate when it comes to the real world. Indeed, the applicability of RL in the real world is held back by data efficiency to reduce training time, limited availability of sensors and safety of both the robot and the surrounding environment.

In this thesis, we evaluate the best strategies to train an RL agent in the real world to control a small scale electric car using only a camera sensor. In particular, we assessed different \textit{representation learning} techniques and \textit{environment reset} strategies, first in a faithful simulated environment and then in the real world.

Finally, we also experimented with \textit{sim2real} (simulation to real) techniques to evaluate whether an RL agent trained in one domain (i.e., simulation) can be transferred into the other (i.e., real).
\vspace{\fill} 
\end{abstract}

\begin{acknowledgements}
\vspace{\fill} 
The writing of this thesis was made possible by Prof. Paolo Tonella, and co-supervisor, Dr. Matteo Biagiola from the Software Institute. I want to thank them for giving me the opportunity to work with them in the subject of my interest and for their immense availability and kindness that has always helped me, event in difficult moments when their advice was fundamental for the success of this project, from the very first to the very last day. 

For having achieved my academic goals I have to thank also all my classmates as I consider them indispensable for my academic career, with them I shared laughter, critical moments and knowledge, we supported each other in all phases of our Master, and without them this wouldn't have been such a great journey.

I also thank my parents Salvatore and Giovanna and my sisters Jessica and Giada, who from outside the university have  contributed in their own way to my life in all these years of study, making it normal and concretely possible.

Last, but definitely not least, I want to thanks the newcomers in my life. My girlfriend Lara who has been supportive since the very first day we met without hesitation and my newly born nephew Amanda. They both have brightened my days contributing to the success of my studies.
\vspace{\fill} 
\end{acknowledgements}


\tableofcontents 
\listoffigures %optional
\listoftables %optional

\mainmatter

\chapter{Introduction}

Reinforcement Learning (RL) is a branch of machine learning that has proven to be a very general framework to learn sequential decision-making tasks modeled as Markov Decision Processes (MDP) \citep{vanOtterlo2012} without the need for labeled data. Under this framework, an RL agent interacts with an environment in discrete time steps, observing the state of the environment and deciding actions based on it, in order to maximize a given reward function to solve a certain task. Such task is usually \textit{episodic}, i.e., there are clear boundaries to determine when the task is achieved.

The first RL algorithms were tabular and could solve simple tasks and games~\cite{Sutton1998}. The advent of \textit{Deep Learning} made it possible for RL algorithms to scale towards interesting and challenging problems, where the state and/or the action space are too big to be stored in tables (e.g. images and continuous actions). Most notably, the first Deep RL (DRL) algorithm, called Deep Q Network, was able to surpass humans in most of the Atari games~\cite{atari}. Recent developments in RL have shown how the framework can deal with even more complex games, from the game of Go~\cite{alphago} to multiplayer games such as Starcraft II~\cite{alphastar} and Dota II~\cite{opeaifive}.

Despite the impressive performance, RL is still confined to simulated environments, where training data can be easily collected by running several instances in parallel and safety is not a primary concern. Indeed, progress towards bringing RL into the real world, has been much slower w.r.t. the progress made in simulation, given the higher complexity and unpredictability of real-world environments. One of the domains in which RL has been applied to the real world is robotics~\cite{smith2022walk,pmlr-v164-raffin22a,gu2017deep}. However, training RL agents to control physical robots in the real world presents several challenges that are absent in simulation. First of all, training cannot be easily parallelized and therefore the RL algorithm needs to be \textit{data-efficient}, i.e., it needs to be able to reuse the collected experience for training since acquiring data is costly. Secondly, the movements of the robot need to be taken into account both to protect the environment around the robot that might carry out unsafe actions during learning and to safeguard the robot itself that can be damaged by the high-frequency actions the robot is controlled with. Moreover, the number of sensors in the real world is limited compared to simulated environments. Having less sensors impacts the design of the reward function that is necessarily more \textit{sparse} (i.e. it gives less guidance to the agent during training) than in simulation. Finally, when the RL agent completes an episode, either successfully or unsuccessfully, a \textit{reset} of the environment is needed, in order to restore its initial state. Depending on the context, such task in the real world might require substantial human intervention.

Positive results in training RL agents to control robots in the real world often come from very constrained applications. For instance, the OpenAI team was able to train a RL agent to control a robotic hand in order to solve the Rubik's cube~\cite{rubik}. Other examples are robotic arms that have been trained to reach a certain object, pull or push a door and to pick an object and place it in a target location~\cite{gu2017deep}. The whole RL training process for such tasks can be almost entirely automated requiring little to no human supervision.

On the other hand, there are fewer tasks that have been addressed with RL in real-world uncontrolled environments~\cite{smith2022walk,DBLP:journals/corr/abs-2008-00715}. In this thesis we target with RL the problem of \textit{self-driving} and, in particular, we are interested in the \textit{lane-keeping} task in which the RL agent needs to control a car to drive along a predetermined track. This problem is usually tackled using the Supervised Learning (SL) approach both in academia and in industry. In such learning paradigm a dataset of driving experience is collected, usually composed of pairs of images and labels. In its simplest form the label is the recorded steering angle applied by the human driver corresponding to a certain image. Then, the dataset is used to train a Deep Neural Network (DNN) to minimize the error between the predicted steering angle and the ground truth label. Even though SL has proven successful for this task in the real world, it has some disadvantages. First, it requires huge labeled datasets that need to be constantly updated to take into account new scenarios. Furthermore, a SL model is trained offline and, as such, it cannot predict the effects that a certain action has on the environment. On the other hand, a RL agent is trained online, thereby being able to learn and adapt based on the effects of its actions, respecting the sequential nature of the task of driving.

%Interesting results in real-world applications comes often from very controlled environments, for example, solving a Rubik's cube requires a limited set of actions, and the reset of the cube to the initial state is very simple. The objective of this thesis is to learn an RL self-driving DonkeyCar, a scale remote-controlled electric car, first in simulation and then replicate the same result in a very uncontrolled real-world environment. The state-of-the-art self-driving cars algorithms, created by Google with Waymo and Tesla, rely on Supervised Learning (SL) techniques such as Convolutional Neural Networks (CNNs) for image processing, feature detection, and extraction, and Recurrent Neural Networks (RNNs) for processing temporal data. Even though supervised learning is very successful in autonomous driving, it does not come at no cost, it requires huge labeled datasets that need to be consistently updated to face new scenarios. Furthermore, SL methods typically are used to generate predictions about the surroundings of the car and upon that decisions are taken and do not take into account that each decision influences future events, which in turn influence future decisions. In other words, they try to imitate data but do not have the consciousness of the real world. RL, on the other hand, allows learning a policy, thereby creating models able to make their own decisions, take actions, react and adapt based on the feedback they receive. 

The objective of this thesis is to train a RL agent to control a physical 1:16 scale radio-controlled car (called DonkeyCar~\footnote{https://www.donkeycar.com/}) and make it drive along a physical track. The car is equipped with a camera, which is the only sensor the agent has to perceive its surroundings. We first addressed the problem in simulation, by using a high-fidelity simulator and a faithful representation of the physical track. We use simulation to carry out experiments to evaluate different training configurations in order to select the best choice when training in the real world. In particular, we experimented with different Representation Learning (ReL) techniques to speed up learning, as learning directly from raw images is known to be ineffective~\cite{DBLP:journals/corr/abs-2008-00715}. Moreover, we evaluated different reward functions and several reset strategies. Then, we transferred this knowledge to the real world and trained a RL agent which is able to learn to drive along the physical track in a reasonable amount of time.

%Since training an autonomous agent from raw images is expensive and has been proven unsuccessful \citep{DBLP:journals/corr/abs-2008-00715}, we first investigate a few Representation Learning (ReL) techniques such as AutoEncoders and Variational AutoEncoders. ReL is a technique designed to extract abstract features from data and reduce their complexity. Furthermore, techniques for a smooth exploration of the environment such as state-dependent exploration are implemented. A reward function that has proven to work in both simulated and real-world environments has been designed even if the sensors are scarce. In particular, as a first step of the experiments, we trained successfully a simulated proof-of-work RL agent that autonomously drives by taking actions based on what a camera sensor, with which it is equipped, sees as unique information about the surroundings. As a second step, the model is successfully replicated in a real-world environment with an investigation of the best training strategy in terms of the starting point of the car which crucially defines the learning success. 

Finally, we also experimented with \textit{sim2real} (simulation to real) techniques that can be used to transfer knowledge from one environment (e.g. in simulation) to another one (e.g. in the real world). Our attempts can be useful to understand the direction of future endeavors at addressing the same problem.

%Finally, a few unsuccessful experiments which need to be explored more are run in a Sim2Real procedure through the use of CycleGan \citep{CycleGAN2017}, where an agent trained in simulation is deployed in the real world and vice-versa, thus leading to cheaper training processes and more reliable benchmarking. 

\chapter{Background}

\section{Reinforcement Learning}

Reinforcement Learning (RL) is a branch of machine learning, alongside supervised learning and unsupervised learning, that defines a set of algorithms meant to learn how to act in a specific environment without the need of labeled data to learn from. 

The algorithm defines the agent that learns a given task, for example, walking, driving and playing a game, by trial and error, while interacting with an environment which can be real or simulated. Whenever the agent makes a set of good actions it receives a positive reward, which makes such actions more likely in the future. State, action and reward are the most important concepts in RL. The state represents the current situation of the environment. If the agent is a humanoid robot and the task is walking, one possible state representation is the positions of its actuated joints. The action set or space in case of continuous domain, describes what the agent can do in a particular state. In the humanoid robot example above, the action space is a n-dimensional vector where each dimension represents the torque command to each of the n joint motors. Finally the reward is a measure of how good are the actions carried out by the agent. 

\begin{wrapfigure}{r}{0.6\textwidth}
  \begin{center}
    \includegraphics[width=0.6\textwidth]{background/rl.png}
  \end{center}
  \caption{Reinforcement Learning loop}
  \label{fig:rl}
\end{wrapfigure}

The reward function, usually human-designed, assigns a score to the action taken by the agent. Every action that leads to a \textit{good} state increases the score and vice-versa. As described in Figure \ref{fig:rl} the agent interacts with the environment in discrete time steps. At time $t$ it gets the current state $s_{t}$ and the associated reward $r_{t}$ then the action $a_{t}$ is chosen from the set of available actions. After receiving the chosen action, the environment moves to a new state $s_{t+1}$ and the reward $r_{t+1}$ is given back to the agent. The total discounted reward (also known as return) to be maximized is:

\begin{equation}
  \label{eq:totalreward}
  R=\sum _{t=0}^{T}\gamma ^{t}r_{t}
\end{equation}
where $T$ is the time horizon (eventually $\infty$), $\gamma \in [0,1)$ is the discount factor which makes future rewards worth less than immediate rewards. The reward function is fundamental to the agent in order to learn and optimize a policy function $\pi$:
\begin{align}
  \displaystyle \pi &:A\times S\rightarrow [0,1] &  \displaystyle \pi (a,s)&=\Pr(a_{t}=a\mid s_{t}=s) \label{eq:label1}
\end{align}
The policy is a mapping that gives the probability of taking action $a$ in state $s$. By following the policy the agent takes the action that maximizes the reward. However, the policy, especially during training, is not deterministic. This is due to one of the fundamental challenges in RL, i.e. the exploration-exploitation dilemma \citep{Sutton1998}. Indeed, the agent needs to repeat the actions it already knows to be rewarding but, at the same time, it needs to explore the environment to discover actions that can lead to an even higher reward. 
The final goal of the algorithm is to learn a policy that maximizes the expected cumulative reward:
\begin{equation}
  \label{eq:rlobj}
  J(\pi)=\mathbb{E}_{\pi}[\sum _{t=0}^{T}\gamma ^{t}r(s_{t},a_{t})]
\end{equation}
The expectation term is added because both the policy and the environments are usually stochastic.
There are multiple ways to learn the optimal policy $\pi^{*}(s)$ assuming the \textit{State Transition probability matrix} $P$ that describes the probability of moving from one state to any successor state is known. The first one is called \textit{Value iteration}, which exploits the state value function $V^{\pi}(s)$ and the action value function $Q^{\pi}(s, a)$. The state value function $V$ is the expected return starting from the state s and following the policy $\pi$:

\begin{equation}
  \label{eq:statevalue}
  V(s) = E_{\pi}[\sum_{t=0}^{T-1} \gamma^t r_t \mid s_t=s]
\end{equation}
while the action value function $Q$ is the expected return starting from the state s and taking action $a$ by following the policy $\pi$ :

\begin{equation}
  \label{eq:actionstatevalue}
  Q(s,a) = E_{\pi}[\sum_{t=0}^{T-1} \gamma^t r_t \mid s_t=s, a_t = a]
\end{equation}
There is an important relationship between the two functions \ref{eq:statevalue} and \ref{eq:actionstatevalue}, in fact they can be written in terms of each other:

\begin{equation}
  \label{eq:statevaluefromQ}
  V(s) = \sum_{a\in A} \pi(a \mid s) Q^{\pi} (s,a)
\end{equation}

\begin{equation}
  \label{eq:actionstatevaluefromV}
  Q(s,a) = \sum_{s'\in S} P(s' \mid s,a) [r(s,a,s') + \gamma V (s')].
\end{equation}
where $P$ is the state transition matrix that gives the probability of reaching the next state $s\textit{'}$ given the state $s$ and action $a$ and $r$ is the reward function that returns the reward value associated with transitioning to the next state $s'$ by taking the action $a$ in state $s$.

In \textit{Value Iteration}, the value function $V$ is randomly initialized and the algorithm, illustrated in Listing \ref{lst:valueiteration}, repeatedly updates the values of $Q$ and $V$ for each state until convergence. When value iteration terminates the functions $Q$ and $V$ are guaranteed to optimal.

\begin{center}
  \begin{minipage}{0.65\linewidth}
    \lstinputlisting[escapeinside={(*}{*)}, caption=Value iteration pseudo code from \citet{Alpaydin:2014:IML:2635955}, captionpos=b, label=lst:valueiteration]{valueiteration.pseudo}
    \end{minipage}
\end{center}
Finally the optimal policy $\pi^{*}$ can be inferred from the optimal $Q^{*}$ function with:
\begin{equation}
  \label{eq:pifromq}
  \pi^{*}(s) = argmax_a Q^{*}(s,a)
\end{equation}
The optimal policy aims at choosing the actions that maximize the optimal $Q$ function all states.

Since the fundamental quantity for the agent is the policy, another way of training the agent is to learn a policy without extracting it from the action-value function $Q$. Therefore, the so-called \textit{Policy Iteration} algorithm seeks to learn the policy directly by updating it at each step as shown in Listing \ref{lst:policyiteration}
\begin{center}
  \begin{minipage}{0.65\linewidth}
    \lstinputlisting[escapeinside={(*}{*)}, caption=Policy iteration pseudo code from \citet{Alpaydin:2014:IML:2635955}, captionpos=b, label=lst:policyiteration]{policyiteration.pseudo}
    \end{minipage}
\end{center}
Policy iteration is also guaranteed to converge to the optimal policy and it often takes less iterations to converge than the value iteration algorithm.

A major problem arises when the \textit{the State Transition Matrix} of the environment is not known to the agent or the number of possible states is too big to be stored in tables, as for example when the state is an image and/or the action space is continuous. Deep RL algorithms use Deep Neural Networks in order to approximate $Q$ and $V$ instead of storing them in tables. Indeed, DNNs can represent states and actions in a compact way thanks to their ability to generalize to unseen data. 

Besides the quantity that needs to be learnt, i.e. the value functions or the policy, RL algorithms are also categorized by the way such quantities are updated. On-Policy methods evaluate and improve the same policy which is being used to select actions. Off-Policy methods can optimize a certain quantity (usually an action value function $Q$) with data coming from any policy. Such methods are typically more efficient than on-policy methods, as they can reuse already collected experience multiple times. In order to reuse previously gathered data, off-policy methods randomly sample training data from the past experience stored in buffers, generally called replay buffer. The replay buffer contains a collection of experience tuples ($s$, $a$, $r$, $s'$), where $s$ is the state, $a$ the action taken in state $s$, $r$ the corresponding scalar reward and $s'$ the new state reached by taking action $a$.

\section{OpenAI Gym Interface}

Gym is an open source library that defines a standard API to handle training and testing of RL agents, while providing a diverse collection of simulated environments. The environment is of primary importance to a RL algorithm since it defines the world the agent lives and operates in. 

\begin{wrapfigure}{r}{0.6\textwidth}
  \begin{center}
    \includegraphics[width=0.6\textwidth]{background/env.png}
  \end{center}
  \caption{DonkeyCar environment}
  \label{fig:gym}
\end{wrapfigure}

The standard interface designed by Gym, makes it easy to interact with environments, both made available by Gym and externally developed. The Gym interface is simple and capable of representing general RL problems. The DonkeyCar environment, shown in Figure \ref{fig:gym} and used in this thesis, is an example of custom environment. With Gym we can define the \textit{ action space} of the car, i.e., all the operations the agent controlling the car can carry out, such as steer and accelerate. Furthermore, we can also define the \textit{observation space}, i.e., what the agent perceives about the environment at each timestep. For instance, a possible observation space for an agent controlling a car is the camera image.

Moreover, the Gym documentation provides a reference template, shown in Listing \ref{lst:gym}, that describes what are the fundamental methods a Gym environment should implement to work properly.
Any existing environment built with Gym implements the following methods:

\begin{center}
  \begin{minipage}{0.45\linewidth}
    \lstinputlisting[caption="Gym template", captionpos=b,  label=lst:gym]{gym.py}
    \end{minipage}
\end{center}

\begin{itemize}
  \item \textbf{init:} every environment should extend the gym.Env class and override the variables \textit{observation\_space} and \textit{action\_space} specifying the type of observations and actions. According to the Gym notation the state of the environment is called \textit{observation} since, in general, the state is not fully observable. Therefore, the observation is the observable part of the state, i.e. what the agent can perceive with its sensors;
  \item \textbf{step:} this method is the primary interface between environment and agent. It takes as input the action to be carried out in the environment and returns a tuple (observation, reward, done) containing the next observation resulting from the action executed in the environment, the corresponding reward value and a boolean flag signaling the end of the \textit{episode}. Indeed, a Gym environment models episodic RL tasks, i.e. finite tasks that terminate when certain conditions hold;
  \item \textbf{reset:} this method resets the environment to its initial conditions returning the initial observation. This method is called to initialize the environment and at the end of each episode, such that the agent starts each episode always with the same initial conditions;
  \item \textbf{render:} this method renders the environment when a parameter \textit{mode='human'} is passed.
  \item \textbf{close:} this method performs any necessary cleanup before closing the environment.
\end{itemize}

Beside the API interface, Gym provides a set of wrappers to modify an existing environment without having to change its underlying code directly. The three main things a wrapper does are:

\begin{itemize}
  \item Transform actions before they are executed in the underlying environment (e.g., ActionWrapper);
  \item Transform observations that are returned by the underlying environment (e.g., ObservationWrapper)
  \item Transform rewards that are returned by the underlying environment (e.g., RewardWrapper)
\end{itemize}

 Furthermore, custom wrappers can be implemented by inheriting from the \textit{Wrapper} class.

\section{Soft Actor Critic - SAC}
The soft actor critic algorithm \citep{art:sac} is a state-of-the-art RL algorithm designed to outperform prior on-policy and off-policy methods in a range of continuous control benchmark tasks. 

It aims to both increase the \textit{sample efficiency} and the robustness of the policy at test time. A poor sample efficiency is typical of on-policy RL methods since they require new sample to be collected for each gradient step. In order to improve the sample efficiency, SAC adopts an off-policy approach where the experience is stored in a replay buffer such that it can be reused multiple times during training. Moreover, SAC is based on the maximum entropy framework which maximizes both the expected return and the entropy of the policy. This objective is expressed in the following equation:

\begin{equation}
  J(\pi)=\mathbb{E}_{\pi}[\sum _{t=0}^{T}\gamma ^{t}r(s_{t},a_{t})+\alpha H(\pi(\cdot \mid s_{t}))]
\end{equation}

where $\alpha$ is a temperature parameter that weighs the entropy term and thus controls the policy stochasticity. Maximizing both the expected reward and the entropy of the policy is beneficial to obtain policies that are robust w.r.t. unexpected situations at testing time. Moreover, training a stochastic policy encourages a wide exploration of the environment, promoting diverse behaviors of the agent.

\section{Generative Adversarial Networks - GAN}

\begin{wrapfigure}{r}{0.6\textwidth}
  \begin{center}
    \includegraphics[width=0.6\textwidth]{background/gan.png}
  \end{center}
  \caption{GAN diagram}
  \label{fig:gan}
\end{wrapfigure}

Generative Adversarial Network is a framework introduced by \citet{art:gan} for training generative models in an unsupervised fashion. GANs can be used, for example, to generate visual paragraph \citep{Liang_2017_ICCV}, realistic text \citep{pmlr-v70-zhang17b}, photographs of human faces \citep{https://doi.org/10.48550/arxiv.1710.10196} and Image-to-Image translation \citep{Isola_2017_CVPR}.
The learning process involves two neural networks that are trained in an adversarial way, i.e. with a contrasting objective. Indeed, as shown in Figure \ref{fig:gan}, the generator G generates inputs (e.g., images) starting from random noise and the discriminator D needs to distinguish whether such inputs belong to the original dataset or not. GANs fall under the branch of unsupervised learning since the training process does not need labeled data as the generator is guided by the discriminator in order to generate inputs that resemble those of the training dataset. The discriminator $D$ is trained to maximize the probability of returning the correct label when given both training examples and examples generated by the generator $G$. At the same time the objective of generator $G$ is to minimize the following loss function:

\begin{equation}
  \label{eq:gloss}
  L_G =log(1-D(G(z)))
\end{equation}

where $z$ is the random noise vector, i.e., the latent vector, $D(G(z))$ is the probability that the generated example $G(z)$ comes from the training dataset (represented by the distribution $p_{x}$ where X is the training dataset and $p_{x}$ represents all the possible images that can be in X), which means that $1-D(G(z))$ is the probability that $G(z)$ does not come from from $p_{x}$. Indeed, the objective of the generator $G$ is to generate examples that are indistinguishable from the training examples for the discriminator $D$.

In particular, in order to learn the generator's distribution $p_{g}$ over the training dataset $X$ such that $p_{g}\approx p_{x}$, the authors define a distribution over latent vectors $p_{z}(z)$, which is mapped into the data space with the generator $G(z, \theta_{g})$. Moreover, the discriminator $D(x;\theta_{d})$, with $x \sim p_{g}$, outputs a single value which estimates the probability of $x$ coming from the distribution $p_{x}$ rather than $p_{g}$. $D$ and $G$ are both differentiable functions represented by a neural network with parameters $\theta_{d}$ and $\theta_{g}$ respectively.

In other words, the discriminator and the generator play a minimax game to optimize the function $V(G,D)$:
\begin{equation}
  \label{eq:ganloss}
  min_G max_D V(G,D) = \mathbb{E}_{x\sim \rho_{data}(x)}[log D(x)] + \mathbb{E}_{z\sim \rho_{z}(z)}[log (1-D(G(z)))]
\end{equation}

\section{CycleGAN}
Image-To-Image translation is a complex task where the goal is to transform an image from one domain to another and vice-versa, as shown in Figure \ref{fig:translation}.

\begin{wrapfigure}{r}{0.5\textwidth}
  \begin{center}
    \includegraphics[width=0.5\textwidth]{background/imagetranslation.png}
  \end{center}
  \caption{Image-to-image translation example \citep{CycleGAN2017}. \matteo{Replace figure with Donkey sim2real}}
  \label{fig:translation}
\end{wrapfigure}

Prior papers have been proposed to translate images from one domain to another, however they often require paired training examples between the domains [\citet{https://doi.org/10.48550/arxiv.1612.00835}, \citet{karakan}].
Such paired datasets can be very expensive or even impossible to gather, as in the case of object transfiguration ($horse \leftrightharpoons zebra$).

Cycle-Consistent Adversarial Networks from \citet{CycleGAN2017} (CycleGAN), aims to solve this problem in an unsupervised fashion. The main goal is to learn, using an adversarial loss, a mapping $G:X\rightarrow Y$, where $X$ and $Y$ are two sets representing different domains, such that the image $G(x) $ with $x\in X$ is indistinguishable from an image $y\in Y$. Since the mapping is highly under-constrained, an inverse mapping $F:Y \rightarrow X$ is introduced, together with a cycle-consistency loss to enforce $F(G(x)) \approx x$ and vice-versa.
To accomplish the goal two discriminators $D_{X}$ and $D_{Y}$ are provided. $D_{X}$ tries to distinguish between examples coming from one domain (i.e. represented by the distribution $p_{x}$) and their translations $F(Y)$ and vice-versa for $D_{Y}$. 
The full objective \ref{eq:cycleganloss} includes the adversarial losses and the cycle-consistency loss to encourage a consistent translation from one domain to the other:

\begin{equation}
  \label{eq:cycleganloss}
  L(G,F,D_{X}, D_{Y}) = L_{GAN}(G, D_{Y},X,Y) + L_{GAN}(F, D_{X},Y,X) + \lambda L_{cyc}(G,F)
\end{equation}

where the loss $L_{GAN}(G, D_{Y},X,Y)$ and $L_{GAN}(F, D_{X},Y,X)$ can be constructed from Equation in \ref{eq:ganloss}
% where the loss $L_{GAN}(G, D_{Y},X,Y)$ is described below and $L_{GAN}(F, D_{X},Y,X)$ can be derived similarly:
% \begin{equation}
% L_{GAN}(G, D_{Y},X,Y) = \mathbb{E}_{y\sim \rho_{data}(y)}[log D_{Y}(y)] + \mathbb{E}_{x\sim \rho_{data}(x)}[log (1-D_{Y}(G(x)))]
% \end{equation}
and the following is the cycle-consistency loss:

\begin{equation}
  L_{cyc}(G,F) =  \mathbb{E}_{x\sim p_{x}}[\left \| F(G(X))-x \right \|_1] + \mathbb{E}_{y\sim p_{y}}[\left \| G(F(y))-y \right \|_1]
\end{equation}

where $\lambda$ is a temperature parameter to define the importance of such loss in Equation \ref{eq:cycleganloss} and $\| \cdot \|_1$ is the L1 norm, i.e. a measure of the distance between vectors.

\section{AutoEncoder and Variational AutoEncoder}

\subsection{AutoEncoder}
AutoEncoders (AEs) are artificial neural networks that fall under the branch of unsupervised learning since they learn efficient encoding into a latent space without the need of a labeled dataset. They are generally used for several purposes, for example, dimensionality reduction, image compression, image denoising, image generation, feature extraction and sentence generation [\citet{doi:10.1126/science.1127647}, \citet{8456308}, \citet{7836672}, \citet{7926714}, \citet{8852155}]. 

Taking as example the case of image dimensionality reduction, an AE is composed of two main parts, an encoder $E$ and a decoder $D$. 
\begin{align}
  E(\phi) &:  X \rightarrow Z & D(\theta) &:  Z \rightarrow X'
\end{align}
where $X = \mathbb{R}^{mxn}$ and $Z = \mathbb{R}^{k }$ for some $m,n,k$ and $k\ll mxn$ to reach the goal of dimensionality reduction. Both encoder and decoder are parametrized functions, with parameters $\phi$ and $\theta$ respectively,
As shown in Figure \ref{fig:ae}, the main goal of the encoder is to learn a mapping of each observation of the dataset $x \in X$ into a latent space of smaller dimensionality.  Since a label is not available, in order to measure the quality of the embedded image into the latent space, the decoder is used to reconstruct the image and then compute the reconstruction loss.

\begin{figure}[h]
  \begin{center}
    \includegraphics[width=0.70\textwidth]{background/ae.png}
  \end{center}
  \caption{AE diagram}
  \label{fig:ae}
\end{figure}

In other words, the encoder maps an image $x\in X$ into the latent space producing ${z=E_{\phi }(x)}$ with $z\in Z$; then $z$ is reconstructed by the decoder to bring it back to the original space ${x'=D_{\theta }(z)}$ with $x'\in X'$. Finally, $x'$ can be used as a label with any distance measure $d(x,x')$. Thus the loss to be minimized is computed as follows:

\begin{equation}
\label{eq:aeloss}
  L(\theta ,\phi ) = d(x_{i},D_{\theta }(E_{\phi }(x_{i})))
\end{equation}

\subsection{Variational AutoEncoder}

Variational AutoEncoders (VAEs) addresses the problem of \textit{sparse localization} of data points into the latent space thus providing a more powerful generative capability then AEs.

\begin{figure}[h]
  \begin{center}
    \includegraphics[width=0.80\textwidth]{background/vae.png}
  \end{center}
  \caption{VAE diagram}
  \label{fig:vae}
\end{figure}

As shown in Figure \ref{fig:vae} only a small change with respect to AEs is introduced, i.e. the encoder instead of mapping samples directly into the latent space it encodes a single input as a distribution (usually a normal distribution) over the latent space. Then the concrete latent vector $z$ is produced by sampling such distribution. 

Specifically, the encoder, starting from an image $x\in X$, produces the parameters ${[\mu_x, \sigma_x]=E_{\phi }(x)}$ such that $z$ is sampled from a normal distribution with such parameters $z \sim \mathcal{N}(\mu_x, \sigma_x)$. Consequently, the decoder brings it back to the original space ${x'=D_{\theta }(z)}$ with $x'\in X'$. Finally, $x'$ can be used as a label with any distance measure $d(x,x')$. Thus the loss to be minimized is computed as follows:

\begin{equation}
\label{eq:vaeloss}
  L(\theta ,\phi) = d(x_{i},D_{\theta }(E_{\phi }(x_{i}))) + KL[\mathcal{N} (\mu_x, \sigma_x),\mathcal{N}(0, 1)]
\end{equation}

where the first term is equivalent to the loss function in Equation \ref{eq:aeloss} and the KL term is the Kullback-Leibler divergence, which is a measure of how a probability distribution is different from another. The KL divergence acts as a regularization term by enforcing predicted distributions to be close to the normal distribution with mean 0 and standard deviation 1, giving to the latent space two main properties: continuity (close points in the latent space should be close also when decoded) and completeness (any point sampled from the latent space should always be meaningful once decoded). 

\section{DonkeyCar} \label{sec:donkeycar}

DonkeyCar, shown in Figure \ref{fig:donkey}, is an open source DIY platform providing software and hardware tools for the development of self-driving car algorithms. The basic car is a simple remote-controlled electric car that can be 3D printed or bought as a kit for an affordable price. The car can be customized with additional sensors as LIDARs and IMUs to provide more information about the surroundings of the car during driving.

\begin{wrapfigure}[19]{r}{0.5\textwidth}
  \begin{center}
    \includegraphics[width=0.5\textwidth]{background/donkey.jpeg}
  \end{center}
  \caption{Assembled DonkeyCar}
  \label{fig:donkey}
\end{wrapfigure}

In particular, the car used for the purposes of this thesis, is a basic DonkeyCar equipped with an 8-megapixel IMX219 sensor that features an 160 degree field of view. It is capable of capturing images with a resolution of 3280x2464 and video recording up to a resolution of 1080p at 30 frames per seconds. In order to process all the information coming from the camera, control the motors and run the self-driving car software, the car is equipped with an NVIDIA Jetson Nano micro-controller. The power comes from a LiPO battery of 11.1V and 2200mAh that runs the electric motor and the micro-controller. Additionally, to expand the operational life of the car, a power-bank can be added to exclusively power the micro-controller, while the LiPO battery is dedicated at powering the engine. 
A DonkeyCar can be remotely controlled either with a joystick or directly by the software.
\chapter{Related works}

When training a Reinforcement Learning model there are several problems that arise, especially when the learning process moves from simulated environments to the real world. In this section a few useful for the purposes of this thesis, are presented.

\section{State Representation Learning} \label{sec:srl}

Reinforcement Learning is a very general method for learning sequential decision making tasks. On the other hand, Deep Learning has become in recent years the best set of algorithms capable of Representation Learning (ReL), i.e. a class of algorithms that are designed to extract abstract features from data. A mix of the two provides a particularly powerful framework for learning state representation, especially when dealing with real world environments that tend to be much more complex and unpredictable than simulated environments. In particular, State Representation Learning (SRL) is a specific type of ReL where extracted features are in low dimension, evolves over time and are affected by agents actions. The low dimensionality allows easier interpretation by humans, it mitigates the curse of dimensionality and it speeds up the policy learning process. Thus, SRL is well suited for Deep Reinforcement Learning applications. \citet{DBLP:journals/corr/abs-1802-04181} presented a complete survey that covers the state-of-the-art on SRL. Feature extraction is a set of algorithms whose objective, as the name suggests, is to decompose a particular data point into smaller identifiable components that are useful for the learning task at hand. Taking as example a dataset of portraits, a set of features that can compose each picture can be the hair color, skin color, face shape and so on. Training a neural network to learn those features may be accomplished by compressing the image into a smaller vector, discarding all the unnecessary information that are not relevant for the learning task where each dimension would represents a feature like the ones just described. However, a feature not necessarily describes a human interpretable aspect of the data, rather it can even lack semantic meaning.. In particular, SRL techniques exploit the time steps, actions, and eventually rewards, to transform observations into representative states, a low dimensionality vector that contains the most relevant features to learn a particular policy. The better the policy or the speed with which it is learned, the more the features extracted are significant to the model.

\section{Improving sample efficiency} \label{sec:sampleefficiency}

In order to define the state of the environment in our experiment we use a camera as described in Section \ref{sec:donkeycar}. However, training a model from high-dimensional images with reinforcement learning is difficult, in previous Section \ref{sec:srl} we described an approach to mitigate those difficulties. In this section we present a specific method that is used for the purposes of this thesis.

Deep convolutional encoders can learn a good representation even though they generally require large amounts of training data. Using off-policy methods and adding an auxiliary task with an unsupervised objective can naturally improve sample efficiency and add stability in optimization but they often lead to suboptimal policies as described in \citet{DBLP:journals/corr/abs-1910-01741}. They revisit the concept of adding an encoder to off-policy RL methods and provide a simple and  effective autoencoder-based off-policy method that can be trained end-to-end. The main focus is in finding the optimal way of training a RL agent using SRL.

In practice, in their experiment, the AE is composed of a convolutional encoder that maps an image observation to a low dimension vector into the latent space and a deconvolutional decoder the reconstruct the latent vector back to the original image. While several auxiliary objectives could be used to improve the learned representation, they target on the most general and widely applicable, an image reconstruction loss avoiding task dependent losses. After that, a SAC algorithm is used to learn some task from the latent state of the environment.

There are two options, the first one seeks to train the agent alternately with the encoder with both kept indipentent from each other. So the AE is pretrained and then a few iteration are used to improve the AE with its own loss, later on, the agent is trained with the encoder kept constant. The algorithm keeps iterating between this two phases until convergence. The second option, seeks to learn a latent representation that is well aligned with the underlying RL objective, thus the AE network is updated with the gradient coming from the actor, critic and the AE itself. However, this attempt of joint representation learning was proven unsuccesful. For this reason, our focus is on the first alternating representation learning. The last thing to define is how often the encoder should be updated. From the tested tasks is evident that it should be updated at the end of every episode, however, even if it is never updated after the first pre-training, the result are still very good. Beside that, an on going update would require more computational power to complete all the algorithm steps in the same amount of time. Since this work aims to solve a real-time problem, it is necessary that a certain number of frames are processed per second that is why the single pretrain is preferable in the context of microcontroller, PC without a GPU and over-the-air communication. Even though this could lead to a slightly longer training, it would speed up the single iteration.

\section{Smooth exploration}

When moving a RL algorithm from a simulated environment to the real world, the unstructured step-based exploration often very successful in simulation, leads to unstable motion patterns. This may results in poor exploration, longer training and even damages to the robot's motors that can be expensive. \citet{pmlr-v164-raffin22a} handle the issue by including a state-dependent exploration (SDE) to current Deep Reinforcement Learning algorithms. In most RL algorithm the standard for exploration is to sample a noise vector from a Gaussian distribution indipendent from the environment and the agent, and then it is added to the controller output. SDE replaces the sampled noise with a state-dependet exploration function. This results in smoother exploration and less variance for each episode. In practice the solutions is as simple as sampling a noise vector as a function of the actual state $s_t$ and adding it to the choosen action.

\section{Learning to Drive - L2D}
Learning to Drive (L2D) \citep{DBLP:journals/corr/abs-2008-00715} is a low-cost benchmark for real world autonomous driving learned through reinforcement learning. Since training this types of RL algorithms can be very expensive due to the nature of trial-and-error learning and the cost of a real car, the benchmark are carried out using a DonkeyCar as described in Section \ref{sec:donkeycar}. The authors also provide the source code in order to let every one implent his own RL algorithm to solve DonkeyCar autonomous driving task which we, for the sake of simplicity, use as a baseline of our experiments. They demonstrate that existing RL
algorithms, like Imitation learning, SAC+VAE and Dreamer, can learn to drive the car from scratch. SAC+VAE is also our choise since it performs the best in terms of High-Speed Control. Beside that, they also show as SAC trained directly from the images is not able to learn, which is why we do not consider this option in our test, insted we focus on the aforementioned State Representation Learnign as they did.
\chapter{Experimental setup}

\section{The track and the environment} \label{sec:track}
The track we used is called USI track, shown in Figure \ref{fig:usitrack}. It strongly resembles one used by \citet{DBLP:journals/corr/abs-2008-00715} in Learning to Drive. We do have a simulated version built in Unity and an actual printed track. The choice fell on this track since it is complete, it includes a straightaway, right turn, left turn, wide turn and very steep turn. Beside that, \citet{DBLP:journals/corr/abs-2008-00715} already proved that the agent can learn on this type of track and the focus of this thesis is more on replicating a real agent learning to self-drive in real world and not creating a new model with a particular feature.

\begin{figure}[h]
    \centering
\begin{minipage}{.5\textwidth}
    \centering
    \includegraphics[width=0.85\textwidth]{setup/usi_track_real.png}
    \captionof{figure}{Real USI track TODO}
    \label{fig:usitrack}
\end{minipage}%
\begin{minipage}{.5\textwidth}
    \centering
    \includegraphics[width=0.95\textwidth]{setup/usi_track.png}
    \captionof{figure}{Simulated USI track}
    \label{fig:usitracksim}
\end{minipage}
\end{figure}

The learning of the agent, as shown later on, is straightforward, except the very steep turn which is considerably harder than the others. This difficulty is due to the limited steering angle of the robot and in the real world the aforementioned adversity is even more marked as well see. 

The default starting line in both tracks is where the Donkey is placed in Figure \ref{fig:usitracksim}. Certainly, in the real track it is an imaginary line that we use as a starting point and of course the laying of the car at each episode beginning cannot be exact but approximate. Beside that, there are a few checkpoints, approximately highlighted with a cross in Figure \ref{fig:usitracksim}, troughout the track that can be used as starting points depending on the learning strategy chosen. The simulator provides the following possibilities:

\begin{itemize}
    \item \textbf{Start:} The episode start always at the starting line.
    \item \textbf{Checkpoint:} The episode start at the latest checkpoint reached during the previous episode.
    \item \textbf{All:} All the checkpoints are used cyclically starting from the starting line and proceeding one by one forward for each episode.
    \item \textbf{Random: } The starting point is chosen randomly, between the available checkpoints, at the beginning of each episode.
\end{itemize}
Furthermore, the simulator let us choose where a lap ends. It can end where the DonkeyCar started or at the starting line.

%TODO: AGGIUNGI referenza punti di partenza.

\section{DonkeyCar}

The real Remote Controlled DonkeyCar is essentially a standard DonkeyCar as described in Section \label{sec:donkeycar}. To recap it is a remote controlled car equipped with a microcontroller NVIDIA Netson Nano and a camera sensor. The RGB pictures are taken at a resolution of $320x240$ and at $20hz$ (20 frames per second), which means the algorithm must finish all iteration steps within $0,05$ seconds otherwise it would skip some frames and the learning or the driving may be compromised by the agent's non-responsiveness. This limitation is present only in real world since the simulator time can be slowed down to meet the needs. In our setting we have a standard 3 cell LiPO battery of 11.1V and 2200 mAh to power just the motors and the controller. During the training in the real world we often noticed a slow regression in term of speeds of the car, iteration after iteration. However, this problem can be solved by disconnecting the LiPO battery for a moment from time to time to restore full speed, which is why we suspect this problem may be caused by the battery. Furthermore, an external power bank with 10000mAh/37Wh of capacity and an output of 5V and 2.4A powers the Jetson Nano which with this capacity, is more than enough to overcome the longevity of the LiPO battery.

\section{Training modality}

\subsection{Simulation}\label{subsec:sim}
In simulation, the training of the agent is straightforward since all the operation are done on the host machine, moreover it is not required a GPU machine to accomplish all the steps in time, at least until the cyclegan is introduced. Even though an on-policy RL algorithm could be implemented, an off-policy algorithm is chosen since in real world, in our setting, the on-policy method is not replicable given the limited computational power of the microcontroller. Beside that, we want to compare the same type of algorithm in the two type of environments. In pratice, a pretrained AE/VAE provides a representation of the state in the form of a latent vector, the agent drives with a policy kept constant during the episode and all the frames and actions are collected into a buffer. When the simulator reports that the car has crashed or went out of track more than a predefined distance, the episode terminates. The episode ends also when the car has reached the starting point or, in our setup, it reaches 1000 steps. Finally, at the end of the episode, a policy is trained using the collected buffer and the new parameters are used to update the driving policy.

\subsection{Real world} \label{subsec:real}
Since the microcontroller equipped by the DonkeyCar is a low capability calculator, a few precautions need to be taken in order to to train the agent. Firstly, as mentioned above, an off-policy method like SAC allows to relocate the actual training, and consequently the very expensive gradient back-propagation, to another machine with more resources. Secondly, the use of representation learning (AE/VAE) allows to reduce significantly the size of the RL neural networks and moreover the pre-training of the encoder can be done in advance on the host machine speeding up the process. In practice the functioning is similar to the one seen in previous Section \ref{subsec:sim}. The microcontroller operates the driving policy, collect the image, forward it through the AE/VAE, then the agent chose an action based on that representation. This process is repated until a human supervisor ends the episode for the car gone off the track. The episode also terminates when the DonkeyCar reaches, in our setup, 1000 steps. Hereafter, all the steps information, like latents vectors, actions and rewards are collected into a buffer up to a predefined size and are sent through the network to the host calculator that actually train the policy at the end of the episode. After the training, the new parameters are sent back to the DonkeyCar and the process is repeated until convergence.

\section{Communication - MQTT}

As described in Section \ref{subsec:real}, when training in real world, the host machine and the DonkeyCar must communicate wirelessly multiple times during the training. MQ Telemetry Transport (MQTT) is the most used messaging protocol for the Internet of Things (IoT). It includes all the rules that define how devices can write (publish) and read (subscribe) data over the internet. The sender (Publisher) and the receiver (Subscriber) communicate via topic and are decoupled from each other. The connection between them is handled by an MQTT broker that filters all incoming messages and distributes them correctly to the Subscribers of the topic. In pratice any device can publish a message on a topic, then the broker take care of dispatching it to subscribers of that topic. In particular we used the HiveMQ broker that allows the connection of up to 100 clients with no cost. The topics defined to manage the communication between the host machine and the DonkeyCar are:

\begin{itemize}
    \item \textbf{Stop car:} The host machine writes a signal on this topic, that is constantly monitorated by the DonkeyCar, to inform that the episode must terminate.
    \item \textbf{Replay buffer:} Once the episode terminate, all the collected information by the DonkeyCar are sent to the host machine through this topic.
    \item \textbf{Replay buffer received:} The host machine uses this topic to acknowledge DonkeyCar that it has received the buffer.
    \item \textbf{Parameters:} Once the training is complete, the host machine sent the updated neural network parameters through this topic.
    \item \textbf{Start episode:} The host machine uses this topic to acknowledge DonkeyCar that a new episode can start.
    \item \textbf{Speed modifier:} This topic can be used by the host machine to inform the DonkeyCar that it must change its throttle by the sent value that can be either positive or negative.
\end{itemize}

Notice that this protocol is not complety reliable so some precautions and check must be done when implenting it, especially in real-time system where some actions cannot be delayed.

\section{Dataset}
With regard to the dataset we need to define two types of dataset. A dataset composed of images collected on the simulator to train the relative autoencoder and after that the simulated RL agent, and a similar one composed of real images collected on the printed track. A few example of each one are respectively shown in Figures \ref{fig:datasetsim} and \ref{fig:datasetreal}.

\begin{figure}[h]
    \begin{minipage}{.33\textwidth}
      \centering
      \includegraphics[width=0.95\textwidth]{setup/s1.jpg}
    \end{minipage}%
    \begin{minipage}{.33\textwidth}
        \centering
        \includegraphics[width=0.95\textwidth]{setup/s2.jpg}
    \end{minipage}%
    \begin{minipage}{.33\textwidth}
        \centering
        \includegraphics[width=0.95\textwidth]{setup/s3.jpg}
    \end{minipage}
    \captionof{figure}{Images extracted from the simulated dataset}
    \label{fig:datasetsim}
  \end{figure}

\begin{figure}[h]
\begin{minipage}{.33\textwidth}
    \centering
    \includegraphics[width=0.95\textwidth]{setup/r1.jpg}
\end{minipage}%
\begin{minipage}{.33\textwidth}
    \centering
    \includegraphics[width=0.95\textwidth]{setup/r2.jpg}
\end{minipage}%
\begin{minipage}{.33\textwidth}
    \centering
    \includegraphics[width=0.95\textwidth]{setup/r3.jpg}
\end{minipage}
\captionof{figure}{Images extracted from the real dataset}
\label{fig:datasetreal}
\end{figure}

During the experiments resulted that a dateset of $\sim 10000$ pictures was enough to reach our goals, furthermore notice that smaller datasets may not be sufficient for the encoder to learn a good representation. To collect each of the datasets are enough $\sim10$ minutes if we run the algorithm at $20hz$ (20 frames per second) as we did. Beside that, all the pictures from the real world must be collected with a certain environmental condition that should remain consistent in time, also during the training of the agent to avoid problems. In our case, it was collected with all windows closed and the with maximum light to make it easy to be replicated. 

Since we want our RL agent to focus exclusively on the track we found convinient to crop the top 100 rows of each pictures to remove the background, and to reduce the complexity of our algorithm, we downscale each images from $320x140$ to $160x80$ before feeding them to the encoder. The resulting pictures are shown in Figures \ref{fig:datasetsimcropped} and \ref{fig:datasetrealcropped}. Note that during training, the training set is split in validation and training set with a ratio $20/80$ and the test set is collected apart and consist of $\sim 1000$ images for each dataset.

\begin{figure}[h]
    \begin{minipage}{.33\textwidth}
      \centering
      \includegraphics[width=0.95\textwidth]{setup/cs1.jpg}
    \end{minipage}%
    \begin{minipage}{.33\textwidth}
        \centering
        \includegraphics[width=0.95\textwidth]{setup/cs2.jpg}
    \end{minipage}%
    \begin{minipage}{.33\textwidth}
        \centering
        \includegraphics[width=0.95\textwidth]{setup/cs3.jpg}
    \end{minipage}
    \captionof{figure}{Examples of cropped simulated images}
    \label{fig:datasetsimcropped}
\end{figure}

\begin{figure}[h]
\begin{minipage}{.33\textwidth}
    \centering
    \includegraphics[width=0.95\textwidth]{setup/cr1.jpg}
\end{minipage}%
\begin{minipage}{.33\textwidth}
    \centering
    \includegraphics[width=0.95\textwidth]{setup/cr2.jpg}
\end{minipage}%
\begin{minipage}{.33\textwidth}
    \centering
    \includegraphics[width=0.95\textwidth]{setup/cr3.jpg}
\end{minipage}
\captionof{figure}{Examples of cropped real images}
\label{fig:datasetrealcropped}
\end{figure}
Finally, for the cyclegan, the dataset can be even smaller $\sim 5000$ pictures for both real and simulation. No crop is applied and a resize to $256x256$ pixels is done to match the network size provided by \citet{CycleGAN2017}. And the test sets match the ones used for the encoders. After applying the cyclegan in our experiments, the pictures are reshaped and cropped to match the need of the autoencoder.
\chapter{Experiments}

\section{AE vs VAE}

Since our main goal is to create an end-to-end RL algorithm composed of an encoder followed by SAC we first need to decide wheter to use an autoencoder or a variational autoencoder. In other words, we want to explore if the stochasticity of a VAE can help in learning a good representation of the actual state. In order to do so, we follow a simple approach, train multiples both AEs and VAEs to compare how much information they are able to recover from the latent vector with an MSE loss on average. As described above, we run a single pre-train on the dataset and then the chosen encoder remains unchanged for the entire duration of the RL agent training. There are varius choices that must be made before proceeding with the RL training. Firts we have to choose between AE and VAE, then the size of the latent vector $z$ and finally wheter to use data augmentation or not to improve the generalization of our model. In particular we consider an AE and a VAE network composed as respectively described in Listings \ref{lst:aenet} and \ref{lst:vaenet}. In Tables \ref{tab:aesim},\ref{tab:aereal}, \ref{tab:vaesim},\ref{tab:vaereal} are shown the result of trainings, each encoder has been trained three times to increase the reliability of the results and the average is reported in the tables. As we see in all cases the encoder performs better when augmentation is disabled, furthermore increasing z\_size to 64 dimensions results in a better reconstruction loss. Finally, the VAE performs slightly better then AE. In figure \ref{fig:realvaeexample} is shown what the reconstructed images looks like for the chosen VAE trained without augmentation and a latent vector size of 64 dimension. All the other encoder reconstructions are shown in APPENDIX TODO.

\begin{figure}[h]
  \begin{minipage}{.50\textwidth}
    \centering
    \includegraphics[height=0.50\textwidth]{experiments/11296.jpg}
  \end{minipage}%
  \begin{minipage}{.50\textwidth}
      \centering
      \includegraphics[width=0.60\textwidth]{experiments/11296.png}
  \end{minipage}
  \captionof{figure}{Real world image processed after cropping with a VAE, z\_size=64 and no augmentation. Reconstruction\_loss=112}
  \label{fig:realvaeexample}
\end{figure}
\begin{figure}
  \begin{minipage}{.50\textwidth}
    \centering
    \includegraphics[height=0.50\textwidth]{experiments/1160.jpg}
  \end{minipage}%
  \begin{minipage}{.50\textwidth}
      \centering
      \includegraphics[width=0.60\textwidth]{experiments/1160.png}
  \end{minipage}
  \captionof{figure}{Simulator image processed after cropping with a VAE, z\_size=64 and no augmentation. Reconstruction\_loss=17}
  \label{fig:simvaeexample}
\end{figure}

\begin{table}
  \centering
  \begin{tabular}{|c|c||c|c|c|c|}
  \hline
  Z\_SIZE & AUGMENTATION & MEAN & STD & MAX & MIN \\ \hline
  \multirow{2}{*}{32} & False & 121.54 & 102.42 & 795.44 & 45.61 \\
  & True & 164.57 & 95.51 & 783.03 & 65.13  \\ \hline
  \multirow{2}{*}{64} & False & 103.54 & 79.14 & 588.14 & 40.84 \\
  & True & 137.24 & 74,02 & 611,81 & 63,05  \\ \hline
  \end{tabular}
  \caption{AE trained in simulation - reconstruction loss}
  \label{tab:aesim}

  \begin{tabular}{|c|c||c|c|c|c|}
  \hline
  Z\_SIZE & AUGMENTATION & MEAN & STD & MAX & MIN \\ \hline
  \multirow{2}{*}{32} & False & 377.07 & 87.53 & 756.7 & 239.46 \\
  & True & 493.84 & 99.40 & 807.67 & 289.99  \\ \hline
  \multirow{2}{*}{64} & False & 311.1 & 78.5 & 695.65 & 177.77 \\
  & True & 411.37 & 77.30 & 647.68 & 241.87 \\ \hline
  \end{tabular}
  \caption{AE trained in real world - reconstruction loss}
  \label{tab:aereal}

  \begin{tabular}{|c|c||c|c|c|c|}
  \hline
  Z\_SIZE & AUGMENTATION & MEAN & STD & MAX & MIN \\ \hline
  \multirow{2}{*}{32} & False & 59.1 & 60.41 & 620.93 & 18.88 \\
  & True & 116.31 & 71.11 & 771.88 & 51.10  \\ \hline
  \multirow{2}{*}{64} & False & 45.15 & 43.49 & 480.22 & 14.34 \\
  & True & 112.17 & 59.79 & 573.19 & 54.28  \\ \hline
  \end{tabular}
  \caption{VAE trained in simulation - reconstruction loss}
  \label{tab:vaesim}

  \begin{tabular}{|c|c||c|c|c|c|}
  \hline
  Z\_SIZE & AUGMENTATION & MEAN & STD & MAX & MIN \\ \hline
  \multirow{2}{*}{32} & False & 227.4 & 44.74 & 418.7 & 140.12 \\
  & True & 263.87 & 52.29 & 478.26 & 172.70 \\ \hline
  \multirow{2}{*}{64} & False & 184.56 & 36.86 & 347.59 & 96.7 \\
  & True & 230.66 & 42.24 & 402.67 & 156.61  \\ \hline
  \end{tabular}
  \caption{VAE trained in real world - reconstruction loss}
  \label{tab:vaereal}
\end{table}


\begin{center}
    \begin{minipage}{0.9\linewidth}
      \lstinputlisting[caption=AE network, label=lst:aenet]{ae.txt}
      \end{minipage}
    \begin{minipage}{0.9\linewidth}
      \lstinputlisting[caption=VAE network, label=lst:vaenet]{vae.txt}
      \end{minipage}
\end{center}

\section{RL algorithm}

\subsection{Reward function}
Designing a reward function that can work on both simulated and real environments is not trivial. In simulation, the environment can provide a better supervision than a human can do. In our real setup we have only the camera frames as states and so no other sensor is available, even though they can be used. In simulation, instead, the environment can tell us how far the donkeycar is from the center line of the track, with this value, we can decrease the reward function by a value proportional to the distance of the car from the center of the roadway as done in Equation \ref{eq:stdreward}. So the reward function is:

\begin{equation}
  \label{eq:stdreward}
    r_t = 1 + throttle\_reward + cte\_penalty + \left\{\begin{matrix}
    if done & crash\_error \\ 
    else & 0  
    \end{matrix}\right.
\end{equation}
where 1 is given for each step taken by the agent in order to incentivize the completion of the largest possible number of steps. The throttle reward incentives the car to go as fast as possible, however in our setup, the throttle is kept constant for the purposes of this thesis. The Cross Track Error (CTE) penalty, is a negative value to incentive the car to stay as close as possibile to the center of the roadway. In case the car exceeds a predefined maximum CTE or crashes, the penalty becomes very high and it is added through the \textit{crash\_error} term. 
This first reward function is used to test our algorithm and after proving it can work, we need to adapt it such that it can work also in real world. To tackle the issue we simply remove the CTE penalty with a negative value to be activated only when an episode ends due to human intervention. The final reward function that has proven to work in both environment and we use in our trainings is:
\begin{equation}
  \label{eq:realreward}
    r_t = 1 + throttle\_reward + \left\{\begin{matrix}
    if done & crash\_error \\ 
    else & 0  
    \end{matrix}\right.
\end{equation}
Since we want the real and the simulated version of our agent as similar as possibile Equation \ref{eq:realreward} is used in both cases.

\subsection{Training the simulated RL agent}
As a baseline for RL algorithm we used the source code provided by \citet{DBLP:journals/corr/abs-2008-00715}. Their algorithm allows both simulated and real training, however training on simulation with communication being over-the-internet is more computationally expensive and more prone to errors. Thus, for the simulation, we refactor the algorithm such that the communication happens locally. Beside that, his algorithm uses an AE which need to be changed with the VAE chosen above. Given that the simulator provides several training modality we test them all to find out which one is the best, to eventually save time in real world. To chose where the car should starts a new lap we trained 4 different model, one for each option provided by the simulator, to identify which one eventually converges and if it does. The quality measures, illustrated below in Figure \ref{fig:agentresults}, are the \textit{Episode success rate} that shows how many laps has been complete on average, the \textit{Episode Reward mean} and the \textit{Episode Length mean}. All the model have been trained for 100k iterations, with a different starting point.\textit{ Agent 1} started each lap at a random checkpoint,\textit{ Agent 2} started always at the starting line,\textit{ Agent 3} at the latest checkpoint reached during the last episode and, finally,\textit{ Agent 4} cyclically uses all the checkpoints.
\begin{figure}[h]
  \centering
  \begin{subfigure}{.5\linewidth}
      \centering
      \includegraphics[width=1\textwidth]{experiments/len_mean.png}
      \caption{Episodes length mean}\label{fig:len}
  \end{subfigure}%
      \hfill
  \begin{subfigure}{.5\linewidth}
      \centering
      \includegraphics[width=1\textwidth]{experiments/rew_mean.png}
      \caption{Episodes reward mean}\label{fig:rew}
  \end{subfigure}
  
  \bigskip
  \begin{subfigure}{.5\linewidth}
    \centering
    \includegraphics[width=1\textwidth]{experiments/success_mean.png}
    \caption{Success rate mean}\label{fig:succ}
  \end{subfigure} 
  \caption{Agents trained in simulation. Each agent has been trained with a different starting modality and has been trained for 100k steps.}
  \label{fig:agentresults}
\end{figure}
During our experiments we noted that, in the best case, a lap may takes $\sim 350$ iterations to be completed. Our agents are all able to successfully learn to drive with a strict maximum CTE except for the agent that started new laps from the latest checkpoint. There are two interesting evidences that come out of those trainings. The first one is that even though the success rate mean approaches $100\%$, that means the agent is able to consistently finish laps, the reward mean keeps growing. This shows a limitation in the reward function used, given that the agent gets a reward for every steps, it learns to finish the lap following the longest path it has discovered. Beside that, the best way to lengthen the path is a zig-zag behavior that allows also a doubling of the reward per lap. Secondly, the agent that start at the latest checkpoint keeps improving the reward up to more than an equivalent completed lap, however it never finishes a lap as described in Figure \ref{fig:succ}. The reason behind this strange behavior is that the agent found a bug in the simulator used, as shown in Figure \ref{fig:bug}. Essentially, there is a a little spot, off track, close to the steepest turn where the CTE is not is correctly detected and consequently the episode is not terminated. The fact this behavior happens only on this agent stands in the training modality. When the agent reach this checkpoint, he cannot easily reach the next checkpoint given the toughness of the turn, instead it finds much easier to explore the bugged spot which is almost right in front of it when it approaches the turn.

\begin{figure}[h]
  \begin{center}
    \includegraphics[width=0.50\textwidth]{experiments/bug.png}
  \end{center}
  \caption{Spotted bug in the simulator}
  \label{fig:bug}
\end{figure}

To furter test the trained agents, for 10 laps it is measured how many times the lap has been completed, how many times the agents crashes, and finally after disabling the CTE threshold, if the agent is still able to recover and finish the lap without crashing. The result are presented in Table \ref{tab:simagent}.

\begin{table}
  \centering
  \begin{tabular}{|c|c|c|c|c|c|}
  \hline
  AGENT & OOT & OBE & LAPS & AVG LENGTH & AVG REW \\ \hline
  1 & 0 & 0 & 10 & 595 & 644 \\
  2 & 0 & 4 & 10 & 599 & 647  \\ \hline
  3 & 0 & 0 & 10 & 624 & 676 \\
  4 & 10 & 29 & 0 & 460 & 495  \\ \hline
  \end{tabular}
  \caption{Agents results averaged over 10 laps. Out Of Track (OOT) measures crashes, Out of Bound Error measure how many times it exceed the max CTE, and finally LAPS counts the completed laps.}
  \label{tab:simagent}
\end{table}

Thus, excluding\textit{ Agent 4 }because of the simulator's bug, three agents learned to drive successfully and in most cases they always stay entirely on track without the need of any additional sensor and with the only problem of the shaky driving which is still acceptable for the purposes of this thesis.

\subsection{Training the real RL agent}
In real world instead the source code provided by \citet{DBLP:journals/corr/abs-2008-00715} is kept untouched beside the encoder, with the main goal being to replicate their results but with a more performing VAE as resulted in our tests. Given that in real world the simulator supervision is not available, all the agents tested in previous section are good candidates to be used, also \textit{Agent 4} that cannot explore anymore the simulator's bug. However from the tests, it resulted that all the agents struggle to learn driving an entire lap in reasonable time, except\textit{ Agent 4 }that start his laps at the latest checkpoint reached in the last episode. Human supervision is about stopping the episode anytime the car exceeds the track boundaries with all 4 wheels, while the host machine automatically stops the car when it reaches 1000 steps ($\sim 2.5$ laps). In figure \ref{fig:realresult} are shown the performances in training. The laying of the car on the latest checkpoint has been intentionally approximate on the area close to the checkpoint. This brought a main advantage, the agent learns quicker since it is able to see the area in front of it from many points of view. It results useful when the car will start to cross many checkpoints per episode, since the direction from which the car arrives to a checkpoint can vary a lot, it has been trained to drive on many possibile trajectories and will join the various sections well. The overall training procedure results to be simple and fast ($\sim$ 30 minutes) to get an entire lap completed. In five to twenty episodes, the first two turn are learned decently. Most of the time is spent on the steepest turn. As shown in Figure \ref{fig:rlen}, the graph is characterized by ups and downs, as soon as the car starts at the starting line, it learns quickly, then, when the steepest curve is reached it struggle to overcome it and the when it enventually does the length start to increase again. The process is repeated until it is almost consistently able to finish a lap. As soon as the agent has learned the steepest turn, it generalizes well on following turn, in fact little time is spent on them.

\begin{figure}[h]
  \centering
  \begin{subfigure}{.5\linewidth}
      \centering
      \includegraphics[width=1\textwidth]{experiments/real_len.png}
      \caption{Episodes length}\label{fig:rlen}
  \end{subfigure}%
      \hfill
  \begin{subfigure}{.5\linewidth}
      \centering
      \includegraphics[width=1\textwidth]{experiments/real_rew.png}
      \caption{Episodes reward}\label{fig:rrew}
  \end{subfigure}
  \caption{Agent trained in real world starting each lap on the latest checkpoint}
  \label{fig:realresult}
\end{figure}

Unfortunately, in real world more metrics to measure the quality of the driving and to make comparisons with the simulated agents are not available. 

\section{Sim to Real}

SimToReal (S2R) or viceversa aims in deploying model trained in one environment to the other. In our case, since the real world environment, in our setup, does not provide enough metrics to benchmark our real world agents, we aim to make it works also in simulation. Another advantage brought by this approach is that an agent trained in simulation, can be moved into the real world and this would result in less expensive training procedures. The idea is to pre-train a CycleGan \citep{CycleGAN2017} for image trasfiguration. In fact, the CycleGan is able to move an image into another domain keeping the original structure unaltered, but applying the style of the other domain. Thus we leverage this property to transform images seen by the simulated camera of the DonkeyCar into what it would see in real world or viceversa. Then, a real agent will eventually be able to drive on the simulator since it does see pseudo-real images. On the other hand, in order to drive a real car with a simulated agent, our DonkeyCar has not enough computational power, hence it could not run in time a CycleGan, that has millions of parameters, in order to make the real car see pseudo-simulated images and drive with the simulated agent. However, the problem can be circumvented by training an agent entirely on simulation but with pseudo-real images.
\chapter{Future work and conclusion}

In conclusion we successfully implemented a Representation Learning (ReL) techniques called Variational AutoEncoder by running several experiments in order to determine which would be the best on our datasets. It allowed us to streamline the reinforcement learning agent training by reducing the sample complexity and actually making it possible in such circumstances, since otherwise it would have been not possible as described by previous works. After defining the ReL technique, before moving to the actual RL training, a reward function that can work on both simulated and real environments was designed. Finally, by putting all the pieces together, both a real and a simulated RL agent were successfully trained by studying which would the best strategies to do it and demonstrating that agents initially trained in simulation can be replicated in very uncontrolled real environments even if with due care. Furthermore, initial studies, which need to be further investigated, to develop a Sim2Real procedure were made through the use of CycleGAN which is able to make a simulated agent see real images and vice-versa. 

As a future work to consolidate the result, besides the further investigation pf the Sim2Real procedure, another reward function to improve the total time an agent takes to complete a lap is needed, a potential solution could be given by:
\begin{equation}
  \label{eq:testreward}
  r_t = - 0.1 + throttle\_reward + cte\_penalty + \left\{\begin{matrix}
    if done & crash\_error \\ 
    else & 0  
    \end{matrix}\right.
\end{equation}
The idea behind this function is that any step gives a negative reward and thus the agent must finish a lap in the smallest number of step to maximize the total episode reward. Minimizing the number steps means also finding the shortest way and consequently reducing the total time spent for a lap. 

Another idea is to equip the car with a few more sensor such as an accelerometer, which would further benefit the training and solve the battery problem of deteriorating performances over time by increasing the throttle when the speed goes down. Besides that, it could help the Donkey reach the cruising speed faster, and minimize the low speed starts at the beginning of each episode.

\appendix %optional, use only if you have an appendix

\chapter{VAE/AE architecture} \label{app:ae/vae}
\begin{center}
    \begin{minipage}{0.9\linewidth}
      \lstinputlisting[caption=AE network, label=lst:aenet]{ae.txt}
      \end{minipage}
\end{center}
\begin{center}
    \begin{minipage}{0.9\linewidth}
      \lstinputlisting[caption=VAE network, label=lst:vaenet]{vae.txt}
      \end{minipage}
\end{center}

\backmatter

% \chapter{Glossary} %optional

%\bibliographystyle{alpha}
%\bibliographystyle{dcu}
\bibliographystyle{plainnat}
\bibliography{biblio}

%\cleardoublepage
%\theindex %optional, use only if you have an index, must use
	  %\makeindex in the preamble

\end{document}